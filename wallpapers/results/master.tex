\documentclass{article}


\usepackage{lscape}
\usepackage{graphicx}
\usepackage{lipsum}
\usepackage{multicol}
\usepackage{tcolorbox}
\usepackage{amsmath}
\usepackage{amsfonts}
\usepackage[paperwidth=9.6in, paperheight=5.4in, left=5mm, right=5mm, top=10mm, bottom=5mm]{geometry}

\newcommand{\N}{\mathbb{N}}
\newcommand{\Z}{\mathbb{Z}}
\newcommand{\Q}{\mathbb{Q}}
\newcommand{\F}{\mathbb{F}}
\newcommand{\R}{\mathbb{R}}
\newcommand{\C}{\mathbb{C}}
\newcommand{\e}{\mathrm{e}}

\newcommand{\norm}[1]{\left|\left|#1\right|\right|}
\newcommand{\abs}[1]{\left|#1\right|}

\definecolor{fg}{RGB}{187, 194, 207}
\definecolor{bg_alt}{RGB}{29, 32, 38}
\definecolor{bg}{RGB}{40, 44, 52}


\begin{document}
  \AddToHook{shipout/background}{%
    \put (0in, -\paperheight){\includegraphics[width=\paperwidth, height=\paperheight]{../wallpaper.png}}%
  }
  \thispagestyle{empty}

  \begin{multicols}{3}
    \begin{tcolorbox}[colback=bg, colframe=bg_alt, coltitle=fg, title=Collatz' Conjecture, fonttitle=\bfseries]
      \color{fg}
      Suppose that $f: \N \to \N$ is defined as:
      \begin{equation*}
        f(n) := \begin{cases}
                  n / 2 & n \in 2\N, \\
                  3n + 1 & \textrm{otherwise.}
                \end{cases}
      \end{equation*}
      and for each $n \in \N$ let $n_{0} := n$ and $n_{k} := f(n_{k-1})$ for $k = 1, 2, \ldots$, then for all $n \in \N$, there exists a $k \in \N$ such that $n_{k} = 1$.
    \end{tcolorbox}

    \begin{tcolorbox}[colback=bg, colframe=bg_alt, coltitle=fg, title=Goldbach's Conjecture, fonttitle=\bfseries]
      \color{fg}
      Let $k \in \N$, such that $k \geq 2$, and $n = 2k$, then $\exists p, q \in \N$, prime, such that $n = p + q$.
    \end{tcolorbox}

    \begin{tcolorbox}[colback=bg, colframe=bg_alt, coltitle=fg, title=Gauss-Bonnet Theorem, fonttitle=\bfseries]
      \color{fg}
      Let $\mathcal{S}$, be a smooth compact surface in $\R^{3}$ and $T$ be a triangulation on $\mathcal{S}$, then
      \begin{equation*}
        \iint_{s} k dA = 2\pi \chi(S; T)
      \end{equation*}
    \end{tcolorbox}

    \begin{tcolorbox}[colback=bg, colframe=bg_alt, coltitle=fg, title=Taylor's Theorem, fonttitle=\bfseries]
      \color{fg}
      Let $f \in C^{\infty}(\R; \R)$, then $\exists s \in \R$ such that:
      \begin{equation*}
        f(x) = \sum^{k}_{n = 0} \frac{f^{(n)}(0)}{n!}x^{n} + \frac{f^{(k + 1)(s)}}{(k + 1)!}x^{k + 1}
      \end{equation*}
    \end{tcolorbox}

    \begin{tcolorbox}[colback=bg, colframe=bg_alt, coltitle=fg, title=Theorema Egregium, fonttitle=\bfseries]
      \color{fg}
      Let $\mathcal{S}_{1}, \mathcal{S}_{2}$, be smooth surfaces in $\R^{3}$, $F: \mathcal{S}_{1} \to \mathcal{S}_{2}$ be a local isometry, and $p \in \mathcal{S}_{1}$ then $k(p) = k(F(p))$.
    \end{tcolorbox}

    \begin{tcolorbox}[colback=bg, colframe=bg_alt, coltitle=fg, title=Lagrange's Theorem, fonttitle=\bfseries]
      \color{fg}
      Suppose $G$ is a finite group, and $H \subseteq G$ a subgroup, then
      \begin{equation*}
        \abs{G}=\abs{G/H}\abs{H}
      \end{equation*}
    \end{tcolorbox}
  \end{multicols}
\end{document}
